% ==============================================================================
%  $RCSfile: intro.tex,v $, $Revision: 1.14 $
%  $Date: 2003-04-18 19:43:26 $
%  $Author: schwiemn $
%
%  Description:
%
%  Last-Ispelled-Revision: 1.9
%
% ==============================================================================


\section{�ber dieses Dokument}

Diese Spezifikation beschreibt die funktionalen und die nicht-funktionalen 
Anforderungen an den IML-Browser GIANT sowie die Rahmenbedingungen, unter 
denen GIANT lauff�hig sein muss.\\
Dieses Dokument ist Grundlage f�r die weitere Entwicklung von GIANT
innerhalb dieses Projekts, das hei�t im Besonderen f�r das Benutzerhandbuch, 
den Entwurf, die Implementierung und den Test des Softwaresystems. 
Das Dokument ist an die Mitglieder der Projektgruppe sowie an den Kunden und
dessen technische Berater gerichtet.\\
Dieses Dokument ist Vertragsbestandteil f�r die weitere Entwicklung von GIANT 
und somit auch Grundlage f�r die Abnahme des fertigen Produktes.


\section{�ber GIANT -- Graphical IML Analysis Navigation Tool}
\index{GIANT}
\index{IML-Browser}

GIANT ist ein Werkzeug, welches an der Universit�t Stuttgart von der 
CodeFabrik@Stuttgart im Rahmen des Studienprojektes A IML-Browser 
des Studiengangs Softwaretechnik entwickelt wird.\\
Ziel dieser Entwicklung ist es, die bereits bestehende HTML-basierte L�sung 
der Bauhaus Reengineering GmbH \index{Bauhaus-Reengineering GmbH}
f�r IML-Graphen durch ein komfortables graphisches Werkzeug zu erg�nzen. 
GIANT soll es dem Kunden erm�glichen, Teile von gro�en IML-Graphen zu
visualisieren. 
Durch die Unterscheidung der verschiedenen Kanten- und Knotenklassen sollen
dar�ber hinaus auch im IML-Graphen enthaltene Analyseergebnisse �bersichtlich
dargestellt werden k�nnen.\\
Das Produkt soll sich durch einen hohen Grad an Wartbarkeit auszeichnen. 
Dies soll insbesondere dadurch erreicht werden, dass die Anbindung an die 
Bauhaus-IML-Graph-Bibliothek derart vorgenommen wird, dass �nderungen in der
Spezifikation des IML-Graphen, wie z.B. das Einbringen neuer Attribute,
m�glichst keine Wartungsarbeiten an GIANT selbst zur Folge haben.
