% ==============================================================================
%  $RCSfile: product.tex,v $, $Revision: 1.44 $
%  $Date: 2003-04-20 15:46:12 $
%  $Author: schwiemn $
%
%  Description: Beschreibung des Produktes. �bersicht �ber die Funktionen
%               und allgemeine Bedienkonzepte.
%
%  Last-Ispelled-Revision: 1.32
%
% ==============================================================================


% ==============================================================================
\section{GIANT Projekte} \index{Projekte} \index{Persistenz}

Innerhalb eines Projektes fasst GIANT Informationen wie
Anzeigefenster und IML-Teilgraphen f�r einen vorgegebenen IML-Graphen
zusammen und speichert diese persistent. Die Zusammenfassung zu
Projekten soll der �bersicht dienen und den Austausch von
Teilergebnissen,
wie z.B. einzelner Anzeigefenster, erleichtern.\\
F�r weitere Informationen zu Projekten siehe Kapitel \ref{GIANT
  Projektverwaltung}.


% ==============================================================================
\section{IML-Teilgraphen und Selektionen}
Hinsichtlich der logischen Zusammenfassung von IML-Knoten und IML-Kanten unterscheidet
GIANT IML-Teilgraphen und Selektionen.

  \subsection{IML-Teilgraphen} \index{IML-Teilgraphen}
  \index{Graph-Knoten} \index{Graph-Kanten} \index{hervorheben} 
  
  Ein IML-Teilgraph ist eine Knoten- und Kantenmenge aus dem
  IML-Graphen mit der Bedingung, dass die Start- und Zielknoten jeder
  Kante Teil der Menge sind. Diese sogenannten Graph-Knoten und
  Graph-Kanten k�nnen in Anzeigefenstern hervorgehoben werden.
  Des weiteren k�nnen die Graph-Knoten und Graph-Kanten von
  IML-Teilgraphen als Fenster-Knoten und Fenster-Kanten in
  Anzeigefenster eingef�gt und layoutet werden. Die IML-Teilgraphen
  sind global in der Anwendung verf�gbar, aber v�llig unabh�ngig von
  den Anzeigefenstern und enthalten insbesondere keine
  Layoutinformationen.

  \subsection{Selektionen} \index{Selektionen}
  Selektionen stellen eine Menge von Fenster-Knoten und Fenster-Kanten
  dar. Selektionen sind immer einem festen Anzeigefenster zugeordnet
  und umfassen nur Fenster-Knoten und Fenster-Kanten des
  Anzeigeinhaltes dieses Anzeigefensters.
  Die Fenster-Knoten und Fenster-Kanten einer Selektion m�ssen keinen
  Teilgraphen bilden, sie d�rfen also auch Fenster-Kanten ohne die
  zugeh�rigen Start- und Zielknoten umfassen.

% ==============================================================================
\section{Anzeigefenster} \index{Anzeigefenster}
Anzeigefenster sind die Fenster von GIANT, in denen eine benutzerdefinierte
Auswahl von Fenster-Knoten und Fenster-Kanten visualisiert wird. 
Es kann beliebig viele
Anzeigefenster geben und jedes Anzeigefenster kann beliebig viele Selektionen
haben.

  \subsection{Pins}\label{Pins} \index{Anzeigefenster!Pins} \index{Pins}
  Da bei gro�en Graphen selten alle zu einem Anzeigeinhalt 
  geh�renden Fenster-Knoten und Fenster-Kanten gemeinsam auf dem Bildschirm
  sichtbar dargestellt werden k�nnen, kann sich der
  Benutzer zu jedem Anzeigefenster eine Liste von Pins anlegen. In den Pins wird
  jeweils die Position des sichtbaren Anzeigeinhaltes und die Zoomstufe
  \index{Zoomstufe}
  gespeichert, so dass zu beliebigen Zeitpunkten die Position des sichtbaren
  Anzeigeinhalts rekonstruiert werden kann.

  \subsection{Visualisierungsstile} \index{Visualisierungsstile}
  Mittels sogenannter Visualisierungsstile kann der Benutzer die
  Darstellung von Fenster-Kanten und Fenster-Knoten auch w�hrend der
  Laufzeit von GIANT beeinflussen.  Da �ber entsprechende XML-Dateien
  verschiedene Visualisierungsstile definiert werden k�nnen, kann
  GIANT so
  an spezifische Problemstellungen angepasst werden.
  Weitere Informationen hierzu sind in Abschnitt \ref{Config
    Visualisierungsstile} zu finden.

\section{Knoten-Annotationen} 
\index{Knoten-Annotationen}

Jeder Knoten kann mit einer textuellen Annotation versehen werden.
Diese Annotation kann in einem Fenster au�erhalb des Anzeigefensters
zur Anzeige gebracht und bearbeitet werden.
F�r weitere Informationen zu diesem Thema siehe Abschnitt \ref{Project Persistenz von
Knoten-Annotationen}.

% ==============================================================================
\section{Anfragen} \index{GSL} \index{Anfragesprache}
Eine vielseitige Anfragesprache -- die GIANT Scripting Language GSL --
stellt nahezu die gesamte Funktionalit�t von GIANT zur Verf�gung und kann 
insbesondere auch zum Aufruf via Kommandozeile
genutzt werden. Die drei Schritte des anschlie�end beschriebenen 
\gq{Drei-Stufen-Konzeptes} k�nnen �ber diese Anfragesprache auch \gq{auf 
einen Schlag} erledigt werden.\\
Die GSL ist unter Kapitel \ref {GIANT Scripting Language} im
Detail spezifiziert.


% ==============================================================================
\section{Drei-Stufen-Konzept} \index{Drei-Stufen-Konzept}

Die im Folgenden beschriebenen drei Schritte sollen dem Benutzer die M�glichkeit
bieten, von den IML-Knoten und IML-Kanten des IML-Graphen ausgehend geeignete
IML-Teilgraphen in Anzeigefenstern zu visualisieren. 
Die Funktionalit�t von GIANT ist so konzipiert, dass zur Visualisierung
von IML-Graphen in Anzeigefenstern diese drei Schritte sequentiell
nacheinander ausgef�hrt werden. Mittels der Anfragesprache GSL 
(siehe \ref{GIANT Scripting Language}) und mittels der �ber die 
GUI zug�nglichen UseCases (beschrieben ab Kapitel \ref{Kapitel-Funktionale Anforderungen})
kann der Benutzer diese Schritte wahlweise einzeln \gq{Step by Step} oder auch 
\gq{auf einen Schlag}, d.h. durch Eingabe einer einzigen Anfrage, ausf�hren.\\
Der \gq{Visualisierungsvorgang} von IML-Graphen ist aber rein logisch betrachtet
in die folgenden drei Schritte unterteilt:

\begin {enumerate}
  \item Erzeugen geeigneter IML-Teilgraphen, d.h.\ Auswahl geeigneter IML-Knoten 
        und IML-Kanten aus dem IML-Graphen mittels der Anfragesprache GSL.
  
  \item Einf�gen dieser IML-Teilgraphen in ein Anzeigefenster unter Anwendung
  eines Layoutalgorithmus. \index{Layoutalgorithmen} 
  
  \item Weitere Bearbeitung der Fenster-Knoten und Fenster-Kanten (wie z.B.
  Verschieben, Annotieren und Erzeugen von Selektionen).

\end {enumerate}
