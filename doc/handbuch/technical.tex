% ==============================================================================
%  $RCSfile: technical.tex,v $, $Revision: 1.6 $
%  $Date: 2003-09-23 19:33:38 $
%  $Author: birdy $
%
%  Description: Hier sind die Anforderungen an die Programmumgebung 
%  spezifiziert. Dazu geh�ren die verwendete Hard- und Software, sowie
%  Schnittstellen zu anderen Produkten.
%
%  Last-Ispelled-Revision:
%
% ==============================================================================

\section{Software}

\index{Anforderungen!an die Software}

Zum Betrieb von GIANT wird folgende Hardware und Software ben�tigt:
\begin{itemize}
  \item Sun Solaris, Linux oder Windows Betriebssystem
  \item Emacs oder vi Texteditor f�r die Anzeige vom Sourcecode
  \item GTK 2.0 oder h�her
  \item GTKAda 2.2
\end{itemize}


\section{Hardware}

\index{Anforderungen!an die Hardware}
Das Programm l�uft auf SPARC Workstations und x86 kompatiblen PCs.
Im Folgenden sind die minimalen Hardwareanforderungen zur Arbeit mit kleinen 
und mittleren Projekten beschrieben. 
Bei gro�en Projekten ist ein Speicherausbau
von 2 GB und mehr empfehlenswert.

\subsection{Hardwareanforderungen SPARC}

\begin{itemize}
  \item UltraSPARC-II 300 MHz
  \item 512 MB Hauptspeicher
  \item 8 Bit Grafik mit einer min. Aufl�sung von 1024*786
  \item Maus mit mindestens zwei Tasten
\end{itemize}

\subsection{Hardwareanforderungen x86}
\begin{itemize}
  \item Pentium III 600 MHz
  \item 512 MB Hauptspeicher
  \item 8 Bit Grafik mit einer min. Aufl�sung von 1024*786
  \item Maus mit mindestens zwei Tasten
\end{itemize}

\section{Installation}

Die Giant-Installation unter Linux geht folgenderma�en vonstatten:\\

Zun�chst mu� das Installationsarchiv (welches sich im Userverzeichnis,
z.B. /home/sysop befindet), entpackt werden. Bei der GIANT-Version
1.1 ist das entpackte Distributions-Archiv ca. 450 MB gro�.\\

Entpacken des gzip-Archives:

�cd /home/sysop�\\

�gzip -d giant-1.1.0.tar.gz�\\

Entpacken des darin befindlichen Tar-Archives:

�tar xf giant-1.1.0.tar�\\

Im nun existierenden Verzeichnis giant-1.1.0 kann
GIANT mit ./giant gestartet werden.

\section{Compilieren}

GIANT kann durch Ausf�hren von �make� im Verzeichnis
src compiliert werden.

