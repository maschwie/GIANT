% ==============================================================================
%  $RCSfile: intro.tex,v $, $Revision: 1.3 $
%  $Date: 2003-09-19 19:50:37 $
%  $Author: birdy $
%
%  Description:
%
%  Last-Ispelled-Revision:
%
% ==============================================================================


\section{�ber dieses Dokument}

Dieses Handbuch beschreibt die Installation und Benutzung des
IML-Browsers GIANT (Graphical IML Analysis Navigation Tool).

\section{�ber GIANT -- Graphical IML Analysis Navigation Tool}
\index{GIANT}
\index{IML-Browser}

GIANT ist ein Werkzeug, welches an der Universit�t Stuttgart von der 
CodeFabrik@Stuttgart im Rahmen des Studienprojektes A IML-Browser 
des Studiengangs Softwaretechnik entwickelt wurde.\\
Ziel dieser Entwicklung war es, die bereits bestehende HTML-basierte L�sung 
der Bauhaus Reengineering GmbH \index{Bauhaus-Reengineering GmbH}
f�r IML-Graphen durch ein komfortables graphisches Werkzeug zu erg�nzen. 
GIANT erm�glicht es dem Kunden, Teile von gro�en IML-Graphen zu
visualisieren. 
Durch die Unterscheidung der verschiedenen Kanten- und Knotenklassen werden
dar�ber hinaus auch im IML-Graphen enthaltene Analyseergebnisse �bersichtlich
dargestellt.\\

\section{Aufbau dieses Dokuments}

In den ersten Kapiteln dieses Handbuches wird beschrieben, wie Anwender
mit GIANT anhand einiger Beispiele Analysen durchf�hren k�nnen.
Dies wird anhand einer Art Tutorials beschrieben. Dies erm�glicht
erfahrenen Anwendern einen schnellen Start (\gq{Top-Down}) mit GIANT.

In den sp�teren Kapiteln wird GIANT Fenster f�r Fenster und Men�punkt
f�r Men�punkt genau erkl�rt (\gq{Bottom-Up}). Dies erm�glicht auch das
Nachschlagen von Details zu einzelnen Funktionen.

