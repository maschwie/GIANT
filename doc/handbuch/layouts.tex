% ==============================================================================
%  $RCSfile: layouts.tex,v $, $Revision: 1.2 $
%  $Date: 2003-09-09 22:28:16 $
%  $Author: birdy $
%
%  Description: Beschreibung der Layouts
%
%  Last-Ispelled-Revision: 1.2
%
% ==============================================================================
%chapter-remark included in spec.tex

\label{Layoutalgorithmen}
Hier werden die Layoutalgorithmen von \product beschrieben, mittels
derer das Layout von Fenster-Knoten innerhalb von Anzeigefenstern
automatisch berechnet werden kann.
\index{Layoutalgorithmen}

\product bietet zwei verschiedene Layoutalgorithmen an:
\begin{enumerate}
  \item Treelayout
  \item Matrixlayout
\end{enumerate}

\section{Treelayout}
\index{Layoutalgorithmen!Treelayout}

Dieser Layoutalgorithmus ordnet Fenster-Knoten baumf�rmig auf dem
Anzeigeinhalt an.

\subsection{Parameter}
\begin{enumerate}
  \item \textbf{Selektion}\\
  Die Selektion, die layoutet werden soll.

  \item \textbf{Wurzelknoten}\\
  Der Benutzer w�hlt den Wurzelknoten des zu erstellenden Baumes
  durch Eingabe der ID des Knotens.
  Sollte die Selektion aus mehreren disjunkten B�umen bestehen,
  so werden die weiteren Wurzelknoten automatisch ermittelt.
 
  \item \textbf{Klassenmengen}\\ 
  Wie auch im Begriffslexikon beschrieben werden 
  Knotenklassen und Kantenklassen zu Klassenmengen zusammengefasst.
  Der Benutzer kann angeben, welche Klassenmengen f�r das Layout
  ber�cksichtigt werden sollen. Der Layoutalgorithmus ber�cksichtigt
  dann nur die Kantenklassen, die zu den gew�hlten Klassenmengen geh�ren.
  Werden keine Klassenmengen vorgegeben, so werden alle Kantenklassen 
  f�r das Layout ber�cksichtigt.

  \item \textbf{Zielposition}\\
Die Position, an die der Mittelpunkt des Wurzelknotens des Baums positioniert werden soll.
\end{enumerate}

\subsection{Beschreibung}
Das Treelayout basiert auf Walkers Algorithmus. Dabei werden folgende Eigenschaften erreicht:
\begin{enumerate}
   \item Alle Kinder eines Fenster-Knotens sind auf der gleichen Ebene
   \item Alle Kinder eines Fenster-Knotens haben den gleichen 
         horizontalen Abstand zueinander
\end{enumerate}

Die Reihenfolge der Kinder von der \texttt{IML\_Reflection}\xspace\"ubernommen. 
In der Grundfunktionalit\"at werden alle Fenster-Kanten betrachtet.

Der vertikale Abstand von Ebene zu Ebene richtet sich nach der H\"ohe des h\"ochsten 
Fenster-Knotens 
(siehe \ref{Visualization-Knoten-Rechteck}).

Falls die \"ubergebene Selektion keinen Baum darstellt, werden die 
Fenster-Knoten in die Ebene  eingegliedert, die bei einer Tiefensuche als 
erstes erreicht wird.

Zerf\"allt die Selektion in nicht-zusammenh\"angende Komponenten, so wird das 
Layout f\"ur jede dieser Komponente durchgef\"uhrt.
Die disjunkten B�ume werden dann horizontal nebeneinander angeordnet.

Als Erweiterung kann der Algorithmus nur bestimmte Fenster-Kanten betrachten. 
Dabei werden im ersten Lauf nur die Fenster-Knoten betrachtet, die durch die gegebenen 
Fenster-Kanten erreicht werden k\"onnen.
Im zweiten Lauf werden Fenster-Knoten, die durch die gegebenen Fenster-Kanten nicht erreicht 
werden konnten, 
im Matrixlayout (siehe \ref{Matrixlayout}) rechts von dem letzten Baum angeordnet.

\section{Matrixlayout}
\label{Matrixlayout}
\index{Layoutalgorithmen!Matrixlayout}


Ein relativ simpler Layoutalgorithmus, der die Fenster-Knoten matrixf�rmig nahe
beieinander auf dem Anzeigeinhalt anordnet.

\subsection{Parameter}
\begin{enumerate}
  \item \textbf{Selektion}\\
Die Selektion, die layoutet werden soll
  \item \textbf{Zielposition}\\
Die Position, an der die linke obere Ecke des Quadrats liegen soll.
\end{enumerate}

\subsection{Beschreibung}
Im Matrixlayout werden die Fenster-Knoten in einem Rechteck ausgerichtet.
Dabei ist die Zahl der Fenster-Knoten pro Zeile und Spalte gleich.  Die Breite
des Rechtecks wird mittels der abgerundeten Wurzel der Anzahl der zu
layoutenden Knoten bestimmt; die H\"ohe ergibt sich automatisch.  Die
Fenster-Knoten werden mittels einer Breitensuche besucht und in dieser
Besuchsreihenfolge in das Rechteck zeilenweise eingef\"ugt. Dabei hat
die obere Begrenzung jedes Fenster-Knotens einer Zeile die selbe Y-Koordinate.
Der vertikale Abstand von Zeile zu Zeile richtet sich nach der H\"ohe
des h\"ochsten Fenster-Knotens (siehe \ref{Visualization-Knoten-Rechteck}).

